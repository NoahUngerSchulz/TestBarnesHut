\documentclass[11pt]{article}

\usepackage{fullpage}
\usepackage{graphicx}
\usepackage{hyperref}
\usepackage{algorithm} 
\usepackage{algpseudocode} 
\usepackage{parskip} 
 



\title{Comparing Barnes Hut Tree Structures with Physical and Computational Metrics}
\author{Noah Unger-Schulz}
\date{} %don't display the current date
\begin{document}

\maketitle

\begin{abstract}

 The challenges of simulating accurate and efficient n body simulations are important for astronomy, astrophysics and generally understanding the universe around us. In this study we tested the affects of multiple parameters used in n body simulations to find which ones had consistent physical accuracy and were fast enoug to run large scale simulations. We tested 3 force selection methods (Direct summation, Binary Tree, and Octree) and 2 integration methods (Runge Kutta of order 4 and LeapFrog Method ) to find which solutions were both accurate and fast. Accuracy was measured using the the 3 conserved quantities in a gravitational system: Energy momentum and angular momentum defining inaccuracy as large variations in the magnitude of measurements. We found that Octree was both nlog(n) efficiency and had higher accuracy than Binary Tree the other most efficient method. We also found surprisingly no difference in accuracy between the Leapfrog and Runge Kutta methods but a slight increase in speed with the Leapfrog method making it the better option for large scale tests.
  
  
  
  
\end{abstract}

\section{Introduction}

The challenge of predicting the paths of stars has been around even before Newton’s universal law of gravitation. The problem is that there is no way to use differential equations to predict even a random system of 3 gravitational bodies. Computers can’t find the exact solution but can easily calculate an approximation of any number of planets that is needed. The basic method to simulate these systems is called the direct summation with Euler's method integrator. This is very inefficient but the easiest method to implement. It comes in two parts. The direct summation method is just a double for loop:

\begin{algorithm}
    \caption{Direct Summation} 
	\begin{algorithmic}[1]
		\For {each Particle p}
			\For {each Particle q}
				\State Attract p to q
			\EndFor
		\EndFor
	\end{algorithmic} 
\end{algorithm}
The other part of this basic method is Euler's method of integration where each particle increases its acceleration and velocity by approximating the each function as a linear function over time in the range of the time step.
\begin{algorithm}
    \caption{Euler's Method} 
	\begin{algorithmic}[1]
	    \Function{Move}{Particle with Velocity, Acceleration and Position}
		    \State Velocity+= Acceleration*dt
		    \State Position+= Velocity*dt
        \EndFunction
	\end{algorithmic} 
\end{algorithm}
\vspace{5mm} %5mm vertical space

These combined algorithms with a small enough value of dt and given enough processing time can accurately predict the position of the particles in any n body simulation but there are far more efficient methods. In this paper we test some of these more efficient methods to try find more accurate and efficient methods to simulate n body systems. In these systems  every N bodies are attracted to N bodies meaning that to find the exact solutions $N^2$ operations are required.
\section{Barnes-Hut Force Methods}



\begin{figure}[h]
\begin{center}
\includegraphics[scale=0.5]{BarnesHut.png}
\end{center}
\caption{
}
\label{setup}
\end{figure}
One important part about gravity simulations is that as the distance between particles increases the force between them quickly approaches zero meaning that the farther away two particles are the less impact they have on each others motion. This fact is taken advantage of in Barnes-Hut simulations to decrease the operations required while having comparable accuracy.

Binary trees are one Barnes-Hut method that we tested.


\begin{algorithm}
    \caption{Binary Tree} 
	\begin{algorithmic}[1]
	\State Box=BoundingBox(Particles) (Step 1)
	\Function{Split}{Box}
	\State Split the box into two smaller boxes each with the same number of particles (Step 2)
	\State Split(Children) (Step 3)
	\State Find the center of mass of the particles in the box 
	\EndFunction
	\State Split(Box)
	\For {each Particle p}
			\For {each Box b}
			\If {p is not in b}
		        \State Attract p to b
		    \EndIf
			\EndFor
		\EndFor
	\end{algorithmic} 
\end{algorithm}
\vspace{40mm} %5mm vertical space

\begin{figure}[h]
\begin{center}
\includegraphics[scale=0.45]{Octree.png}
\end{center}
\caption{
}
\label{setup}
\end{figure}

The octree implementation works similarly but there is a difference instead of splitting the boxes into two it splits them into eighths (hence octree) and instead of splitting the boxes by the number of particles it splits them spatially so each box is the exact same size. 
Here is an example of the 2d equivalent (a quadtree).

\begin{algorithm}
    \caption{Binary Tree} 
	\begin{algorithmic}[1]
	\For {each Particle p}
			\For {each Box b that p is in}
			Parent= the Parent of b
			\For {each child of parent c}
			\If {p is not in c}
		        \State Attract p to c
		    \EndIf
			\EndFor
			\EndFor
		\EndFor
	\end{algorithmic} 
\end{algorithm}
\vspace{50mm} %5mm vertical space


\section{Results}

\begin{figure}[h]
\begin{center}
\includegraphics[scale=0.45]{RunTime.png}
\end{center}
\caption{
}
\label{setup}
\end{figure}
Looking at the Run Time versus number of particles graphs reinforces the difference in efficiency. The graph shows that both the Binary tree and Octree methods are roughly linear but their efficiencies are in fact of order nlog(n) ( denoted $\mathcal{O}(n\log{}n)$ ) compared to the direct summation method which is clearly $\mathcal{O}(n^2)$. The order of a function is defined by the dominant function at extremes. This means as the amount of particles increases the Barnes-Hut simulations will always out compete the direct summation method.

\begin{figure}[h]
\begin{center}
\includegraphics[scale=0.5]{Accuracy.png}
\end{center}
\caption{ 
}
\label{setup}
\end{figure}
We then compared the accuracy of the functions and found that the binary tree method was far less accurate Showing larger variation in Momentum, Angular Momentum and Energy. These large variations indicate that this simulation is not following the conservation laws of all simulations. The octree method on the other hand even outperformed the direct summation method on some of the tests especially those dealing with energy conservation. This may be explained by the fact that the binary tree stores much less spatial information which is the basis on why partitioning keeps consistent approximates . It can also be attributed to the fact that the trees are larger thus can store more information for higher accuracy.

\begin{figure}[h]
\begin{center}
\includegraphics[scale=0.4]{DifVTime.png}
\end{center}
\caption{ We also found that the octree method better approximated the direct summation method.
}
\label{setup}
\end{figure}
\vspace{0mm} 


We also tested two different integration methods Leapfrog and Runge Kutta method. Runge Kutta calculates 4 separate slopes that are then averaged together giving it an order of $\mathcal{O}(dt^4)$ making it known for its accuracy. On the other hand Leapfrog is far less accurate with an order of $\mathcal{O}(dt^2)$. The power of Leap frog is that it is time reversible. This reversibility means that it is symmetric over time leading to less drift in accuracy and hopefully more accuracy over time. 

\begin{figure}[h]
\begin{center}
\includegraphics[scale=0.8]{RK4VLeapfrog.png}
\end{center}
\caption{
}
\label{setup}
\end{figure}
\vspace{100mm}
The main difference between these two different methods was actually just the runtime as RK4 was more efficient compared to LeapFrog. This difference in integration methods is pronounced more in the tree methods than the direct summation so LeapFrog may still be more efficient for direct summation simulations. Surprisingly the famous difference in accuracy was virtually unnoticeable. Even for very large values of dt. Still with the decrease in efficiency was notable meaning that Runge Kutta should be used for more efficient simulations.

\begin{figure}[h]
\begin{center}
\includegraphics[scale=0.5]{MomentumVdt.png}
\end{center}
\caption{
}
\label{setup}
\end{figure}
\vspace{50mm}
We also tested the effects of the length of time steps on the accuracy of the simulation and we discovered this oscillatory motion of the graph. This happens because of the orbital motion of the particles. As the time step decreases the number of cycles that can be completed in the testing period decreases. Eventually with a small enough time step each value converges to a single point. The speed at which this converges may be another indicator of an accurate simulation. The Octree method converges much faster than the Binary tree method indicating it may be more accurate even with larger values of dt.

\begin{figure}[h]
\begin{center}
\includegraphics[scale=0.5]{Density.png}
\end{center}
\caption{
}
\label{setup}
\end{figure}
\vspace{90mm}
Finally to eliminate bias we made sure to test how the density of the particles affected the accuracy of the simulation. As shown by the difference versus density graph the binary tree gets even less accurate as the density increases while the octree method seemingly converged to a small difference from the direct summation method.
\section{Discussion}




In conclusion the Octree method is a good medium between the hyper accurate direct summation and the very quick binary tree method. It has the faster order of runtime for large number of particles it is very accurate and closely approximates the direct summation in almost all cases. It is also more resilient to changes in length of time step and particle density making it the better option for the range of parameters tested. The best integrator that it could be coupled with is the Runge kutta integrator it was shown to have around the same accuracy as the leapfrog method but with a slightly faster runtime. 



\section{Acknowledgements}

I would like to thank both my mentors for their support. Thanks to Owen Young for donating his initial code for me add to and Johnny Powell for helping pick a reasonable project that was inside the scope of the of the time given. Thank you to Maria de Hoyos, Mark Galassi and the entire institute for research in computing for giving me this amazing opportunity and so much support through out the entire process. 

\begin{thebibliography}{1}

   
   \url{https://www.haroldserrano.com/blog/visualizing-the-runge-kutta-method}
   
   \url{http://www.physics.drexel.edu/~steve/Courses/Comp_Phys/Integrators/leapfrog/}


  \end{thebibliography}

\end{document}
